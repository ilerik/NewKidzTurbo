\documentclass[10pt,a4paper]{article}
\usepackage[utf8]{inputenc}
\usepackage[russian,english]{babel}
\usepackage[OT1]{fontenc}
\usepackage{amsmath}
\usepackage{amsfonts}
\usepackage{amssymb}
\usepackage{graphicx}

\usepackage[backend=bibtex,
style=numeric,
%style=alphabetic
%style=reading
sorting=ynt]{biblatex}
\addbibresource{references.bib}

\author{Ilya Eriklintsev}
\title{2D Tests descripition}
\begin{document}
\section{Saltzman's problem} 
Completely taken from \cite{Ni2013}.

"This is a difficult test case to validate numerical schemes with moving boundary and has been extensively studied in the literature \cite{Dukowicz1992}. The set-up of the problem is the following. On a computational domain $ 1.0 \times 0.1$, $100 \times 10$ grid points are distributed in the following (x, y) locations,
\[
 \begin{cases}
   x_{ij} = (i-1) \Delta x + (11 - j) sin(\frac{\pi (i-1)}{100}) \Delta y \\
   y_{ij} = (j-1) \Delta y
  \end{cases}
\]
where $\Delta x = \Delta y = 0.01$.

A gas with specific heat ratio $\gamma = 5/3 $ is filled stationarily inside
the computational domain initially. Then, the left boundary, such as a piston, is moving into the gas with constant velocity $1.0$. Thus, a strong shock wave is generated from the moving piston. On the upper and lower boundaries, a reflection boundary condition is used. This problem has the exact solution. At $t = 0.6$, the shock is expected to be located at $x = 0.8$ with the post-shock density $\rho = 0.6$, velocity $u = 1.0$, and pressure $p = 1.333$. 

In this case, we simply take $U_{g,y} = 0$. The numerical results by the current moving mesh method are shown in Figs. 8-13. The initial mesh is given in Figs. 8 and 9 is the mesh at time $t = 0.6$. The computed pressure, velocity and density are presented in Figs. 12 and 11 along the line y = 0.05. We find that the current numerical results are more accurate in comparison with other scheme, such as in [36,37].In Fig. 13, the mesh and density contour at time $t = 0.9$ are shown."

So we have following problem parameter values for ideal gas inside the domain:
\[
  \begin{cases}
    \rho_{0} = 1 \\
    u_{0} = 0 \\
    e_{0} = 10^-4
  \end{cases}
\]
and left border velocity is set to be $u_{L} = 1.0$.

\medskip
 
\printbibliography

\end{document}