\documentclass[10pt,a4paper]{article}
\usepackage[utf8]{inputenc}
\usepackage[russian]{babel}
\usepackage[OT1]{fontenc}
\usepackage{amsmath}
\usepackage{amsfonts}
\usepackage{amssymb}
\usepackage{graphicx}

\usepackage[backend=bibtex,
style=numeric,
%style=alphabetic
%style=reading
sorting=ynt]{biblatex}
\addbibresource{references1D.bib}

\author{Ilya Eriklintsev}
\title{One dimensional test cases}
\begin{document}
\section{Riemann problem tests}
Five Riemann problems are selected to test the performance of the Riemann
solver and the influence of the initial guess for pressure. The tests are
also used to illustrate some typical wave patterns resulting from the solution
of the Riemann problem. Table 4.1 shows the data for all five tests in terms
of primitive variables. In all cases the ratio of specific heats is ? = 1.4. The
source code for the exact Riemann solver, called HE-E1RPEXACT, is part of
the library NUMERICA [519]; a listing is given in Sect. 4.9.
Test 1 is the so called Sod test problem [453]; this is a very mild test and
its solution consists of a left rarefaction, a contact and a right shock. Fig.
4.7 shows solution profiles for density, velocity, pressure and specific internal
energy across the complete wave structure, at time t = 0.25 units. Test 2,
called the 123 problem, has solution consisting of two strong rarefactions and
a trivial stationary contact discontinuity; the pressure p? is very small (close
to vacuum) and this can lead to difficulties in the iteration scheme to find p?
numerically. Fig. 4.8 shows solution profiles. Test 2 is also useful in assessing
the performance of numerical methods for low density flows, see Einfeldt et.
al. [182]. Test 3 is a very severe test problem, the solution of which contains a
left rarefaction, a contact and a right shock; this test is actually the left half
of the blast wave problem of Woodward and Colella [584], Fig. 4.9 shows solution
profiles. Test 4 is the right half of the Woodward and Colella problem; its
solution contains a left shock, a contact discontinuity and a right rarefaction,
as shown in Fig. 4.10. Test 5 is made up of the right and left shocks emerging
from the solution to tests 3 and 4 respectively; its solution represents the collision
of these two strong shocks and consists of a left facing shock (travelling
very slowly to the right), a right travelling contact discontinuity and a right
travelling shock wave. Fig. 4.11 shows solution profiles for Test 5.
\section{Arbitrary Lagrange-Euler tests}
Current section composed of various tests that used for ALE codes validation.

\medskip
 
\printbibliography

\end{document}